% Options for packages loaded elsewhere
\PassOptionsToPackage{unicode}{hyperref}
\PassOptionsToPackage{hyphens}{url}
\PassOptionsToPackage{dvipsnames,svgnames,x11names}{xcolor}
%
\documentclass[
  letterpaper,
  DIV=11,
  numbers=noendperiod]{scrartcl}

\usepackage{amsmath,amssymb}
\usepackage{iftex}
\ifPDFTeX
  \usepackage[T1]{fontenc}
  \usepackage[utf8]{inputenc}
  \usepackage{textcomp} % provide euro and other symbols
\else % if luatex or xetex
  \usepackage{unicode-math}
  \defaultfontfeatures{Scale=MatchLowercase}
  \defaultfontfeatures[\rmfamily]{Ligatures=TeX,Scale=1}
\fi
\usepackage{lmodern}
\ifPDFTeX\else  
    % xetex/luatex font selection
\fi
% Use upquote if available, for straight quotes in verbatim environments
\IfFileExists{upquote.sty}{\usepackage{upquote}}{}
\IfFileExists{microtype.sty}{% use microtype if available
  \usepackage[]{microtype}
  \UseMicrotypeSet[protrusion]{basicmath} % disable protrusion for tt fonts
}{}
\makeatletter
\@ifundefined{KOMAClassName}{% if non-KOMA class
  \IfFileExists{parskip.sty}{%
    \usepackage{parskip}
  }{% else
    \setlength{\parindent}{0pt}
    \setlength{\parskip}{6pt plus 2pt minus 1pt}}
}{% if KOMA class
  \KOMAoptions{parskip=half}}
\makeatother
\usepackage{xcolor}
\setlength{\emergencystretch}{3em} % prevent overfull lines
\setcounter{secnumdepth}{-\maxdimen} % remove section numbering
% Make \paragraph and \subparagraph free-standing
\makeatletter
\ifx\paragraph\undefined\else
  \let\oldparagraph\paragraph
  \renewcommand{\paragraph}{
    \@ifstar
      \xxxParagraphStar
      \xxxParagraphNoStar
  }
  \newcommand{\xxxParagraphStar}[1]{\oldparagraph*{#1}\mbox{}}
  \newcommand{\xxxParagraphNoStar}[1]{\oldparagraph{#1}\mbox{}}
\fi
\ifx\subparagraph\undefined\else
  \let\oldsubparagraph\subparagraph
  \renewcommand{\subparagraph}{
    \@ifstar
      \xxxSubParagraphStar
      \xxxSubParagraphNoStar
  }
  \newcommand{\xxxSubParagraphStar}[1]{\oldsubparagraph*{#1}\mbox{}}
  \newcommand{\xxxSubParagraphNoStar}[1]{\oldsubparagraph{#1}\mbox{}}
\fi
\makeatother

\usepackage{color}
\usepackage{fancyvrb}
\newcommand{\VerbBar}{|}
\newcommand{\VERB}{\Verb[commandchars=\\\{\}]}
\DefineVerbatimEnvironment{Highlighting}{Verbatim}{commandchars=\\\{\}}
% Add ',fontsize=\small' for more characters per line
\usepackage{framed}
\definecolor{shadecolor}{RGB}{241,243,245}
\newenvironment{Shaded}{\begin{snugshade}}{\end{snugshade}}
\newcommand{\AlertTok}[1]{\textcolor[rgb]{0.68,0.00,0.00}{#1}}
\newcommand{\AnnotationTok}[1]{\textcolor[rgb]{0.37,0.37,0.37}{#1}}
\newcommand{\AttributeTok}[1]{\textcolor[rgb]{0.40,0.45,0.13}{#1}}
\newcommand{\BaseNTok}[1]{\textcolor[rgb]{0.68,0.00,0.00}{#1}}
\newcommand{\BuiltInTok}[1]{\textcolor[rgb]{0.00,0.23,0.31}{#1}}
\newcommand{\CharTok}[1]{\textcolor[rgb]{0.13,0.47,0.30}{#1}}
\newcommand{\CommentTok}[1]{\textcolor[rgb]{0.37,0.37,0.37}{#1}}
\newcommand{\CommentVarTok}[1]{\textcolor[rgb]{0.37,0.37,0.37}{\textit{#1}}}
\newcommand{\ConstantTok}[1]{\textcolor[rgb]{0.56,0.35,0.01}{#1}}
\newcommand{\ControlFlowTok}[1]{\textcolor[rgb]{0.00,0.23,0.31}{\textbf{#1}}}
\newcommand{\DataTypeTok}[1]{\textcolor[rgb]{0.68,0.00,0.00}{#1}}
\newcommand{\DecValTok}[1]{\textcolor[rgb]{0.68,0.00,0.00}{#1}}
\newcommand{\DocumentationTok}[1]{\textcolor[rgb]{0.37,0.37,0.37}{\textit{#1}}}
\newcommand{\ErrorTok}[1]{\textcolor[rgb]{0.68,0.00,0.00}{#1}}
\newcommand{\ExtensionTok}[1]{\textcolor[rgb]{0.00,0.23,0.31}{#1}}
\newcommand{\FloatTok}[1]{\textcolor[rgb]{0.68,0.00,0.00}{#1}}
\newcommand{\FunctionTok}[1]{\textcolor[rgb]{0.28,0.35,0.67}{#1}}
\newcommand{\ImportTok}[1]{\textcolor[rgb]{0.00,0.46,0.62}{#1}}
\newcommand{\InformationTok}[1]{\textcolor[rgb]{0.37,0.37,0.37}{#1}}
\newcommand{\KeywordTok}[1]{\textcolor[rgb]{0.00,0.23,0.31}{\textbf{#1}}}
\newcommand{\NormalTok}[1]{\textcolor[rgb]{0.00,0.23,0.31}{#1}}
\newcommand{\OperatorTok}[1]{\textcolor[rgb]{0.37,0.37,0.37}{#1}}
\newcommand{\OtherTok}[1]{\textcolor[rgb]{0.00,0.23,0.31}{#1}}
\newcommand{\PreprocessorTok}[1]{\textcolor[rgb]{0.68,0.00,0.00}{#1}}
\newcommand{\RegionMarkerTok}[1]{\textcolor[rgb]{0.00,0.23,0.31}{#1}}
\newcommand{\SpecialCharTok}[1]{\textcolor[rgb]{0.37,0.37,0.37}{#1}}
\newcommand{\SpecialStringTok}[1]{\textcolor[rgb]{0.13,0.47,0.30}{#1}}
\newcommand{\StringTok}[1]{\textcolor[rgb]{0.13,0.47,0.30}{#1}}
\newcommand{\VariableTok}[1]{\textcolor[rgb]{0.07,0.07,0.07}{#1}}
\newcommand{\VerbatimStringTok}[1]{\textcolor[rgb]{0.13,0.47,0.30}{#1}}
\newcommand{\WarningTok}[1]{\textcolor[rgb]{0.37,0.37,0.37}{\textit{#1}}}

\providecommand{\tightlist}{%
  \setlength{\itemsep}{0pt}\setlength{\parskip}{0pt}}\usepackage{longtable,booktabs,array}
\usepackage{calc} % for calculating minipage widths
% Correct order of tables after \paragraph or \subparagraph
\usepackage{etoolbox}
\makeatletter
\patchcmd\longtable{\par}{\if@noskipsec\mbox{}\fi\par}{}{}
\makeatother
% Allow footnotes in longtable head/foot
\IfFileExists{footnotehyper.sty}{\usepackage{footnotehyper}}{\usepackage{footnote}}
\makesavenoteenv{longtable}
\usepackage{graphicx}
\makeatletter
\newsavebox\pandoc@box
\newcommand*\pandocbounded[1]{% scales image to fit in text height/width
  \sbox\pandoc@box{#1}%
  \Gscale@div\@tempa{\textheight}{\dimexpr\ht\pandoc@box+\dp\pandoc@box\relax}%
  \Gscale@div\@tempb{\linewidth}{\wd\pandoc@box}%
  \ifdim\@tempb\p@<\@tempa\p@\let\@tempa\@tempb\fi% select the smaller of both
  \ifdim\@tempa\p@<\p@\scalebox{\@tempa}{\usebox\pandoc@box}%
  \else\usebox{\pandoc@box}%
  \fi%
}
% Set default figure placement to htbp
\def\fps@figure{htbp}
\makeatother

\KOMAoption{captions}{tableheading}
\makeatletter
\@ifpackageloaded{caption}{}{\usepackage{caption}}
\AtBeginDocument{%
\ifdefined\contentsname
  \renewcommand*\contentsname{Table of contents}
\else
  \newcommand\contentsname{Table of contents}
\fi
\ifdefined\listfigurename
  \renewcommand*\listfigurename{List of Figures}
\else
  \newcommand\listfigurename{List of Figures}
\fi
\ifdefined\listtablename
  \renewcommand*\listtablename{List of Tables}
\else
  \newcommand\listtablename{List of Tables}
\fi
\ifdefined\figurename
  \renewcommand*\figurename{Figure}
\else
  \newcommand\figurename{Figure}
\fi
\ifdefined\tablename
  \renewcommand*\tablename{Table}
\else
  \newcommand\tablename{Table}
\fi
}
\@ifpackageloaded{float}{}{\usepackage{float}}
\floatstyle{ruled}
\@ifundefined{c@chapter}{\newfloat{codelisting}{h}{lop}}{\newfloat{codelisting}{h}{lop}[chapter]}
\floatname{codelisting}{Listing}
\newcommand*\listoflistings{\listof{codelisting}{List of Listings}}
\makeatother
\makeatletter
\makeatother
\makeatletter
\@ifpackageloaded{caption}{}{\usepackage{caption}}
\@ifpackageloaded{subcaption}{}{\usepackage{subcaption}}
\makeatother

\usepackage{bookmark}

\IfFileExists{xurl.sty}{\usepackage{xurl}}{} % add URL line breaks if available
\urlstyle{same} % disable monospaced font for URLs
\hypersetup{
  pdftitle={Quarto-Jupyter-Matlab-Everything guide},
  pdfauthor={Me},
  colorlinks=true,
  linkcolor={blue},
  filecolor={Maroon},
  citecolor={Blue},
  urlcolor={Blue},
  pdfcreator={LaTeX via pandoc}}


\title{Quarto-Jupyter-Matlab-Everything guide}
\author{Me}
\date{2024-12-18}

\begin{document}
\maketitle


\section{Here's some heading}\label{heres-some-heading}

\begin{enumerate}
\def\labelenumi{\arabic{enumi}.}
\tightlist
\item
  Here's the first point of some listing
\item
  And this is the second point of this pointless listing
\end{enumerate}

After this, there is some text with funny words. With a backslash or two
spaces you can create a line break, as seen here.\\
Then the text continues in the next line.

\begin{itemize}
\tightlist
\item
  some bulletpoint
\item
  more bulletpoints

  \begin{itemize}
  \tightlist
  \item
    funny indented bullet point
  \end{itemize}
\end{itemize}

\begin{Shaded}
\begin{Highlighting}[]
\CommentTok{\# This is a cell for some calculations or plot or something, it\textquotesingle{}s just plain python}
\ImportTok{import}\NormalTok{ numpy }\ImportTok{as}\NormalTok{ np }
\ImportTok{import}\NormalTok{ matplotlib.pyplot }\ImportTok{as}\NormalTok{ plt }

\NormalTok{f }\OperatorTok{=}\NormalTok{ np.linspace(}\DecValTok{0}\NormalTok{,}\DecValTok{1}\NormalTok{,}\DecValTok{1000}\NormalTok{)}
\NormalTok{x }\OperatorTok{=}\NormalTok{ np.sin(}\DecValTok{2}\OperatorTok{*}\NormalTok{np.pi}\OperatorTok{*}\NormalTok{f)}

\NormalTok{plt.close(}\StringTok{\textquotesingle{}all\textquotesingle{}}\NormalTok{)}
\NormalTok{plt.figure(}\DecValTok{1}\NormalTok{)}
\NormalTok{plt.plot(f,x)}
\NormalTok{plt.title(}\StringTok{\textquotesingle{}Some sine function\textquotesingle{}}\NormalTok{)}
\NormalTok{plt.xlabel(}\StringTok{\textquotesingle{}Time\textquotesingle{}}\NormalTok{)}
\NormalTok{plt.ylabel(}\StringTok{\textquotesingle{}Amplitude\textquotesingle{}}\NormalTok{)}
\NormalTok{plt.grid()}
\end{Highlighting}
\end{Shaded}

\pandocbounded{\includegraphics[keepaspectratio]{guide_files/figure-pdf/cell-2-output-1.png}}

Then, by pressing Crtl+Shift+K (just as in a default quarto document)
you can create a quarto output in the format specified in the header. It
depicts everything from Text to code to figures, depending on the format
chosen.

You can even insert funny formulae, no need to change the cell to latex
\begin{equation}
    Y[k] = \sum_{n = -\infty}^{\infty}x[k]\cdot h[k-n].
\end{equation}

If you want to have a nice figure:

\begin{figure}[H]

{\centering \pandocbounded{\includegraphics[keepaspectratio]{recources/example_image.jpeg}}

}

\caption{\center I want this era of artstlye back :(}

\end{figure}%

\subsection{Matlab calls or data calls in
general}\label{matlab-calls-or-data-calls-in-general}

In the following, calls are made to the matlab files for first order
modulator. All the data from simulation are stored in a seperate folder
and can be accessed by just giving relative paths for further processing
in python

\subsection{Matlab calls or data calls in
general}\label{matlab-calls-or-data-calls-in-general-1}

In the following, calls are made to the matlab files for first order
modulator.\\
THE DATA MUST BE EXPORTED BY RUNNING THE MATLAB SCRIPTS MANUALLY.\\
All the data from simulation are stored in a seperate folder and can be
accessed by just giving relative paths for further processing in python

\begin{Shaded}
\begin{Highlighting}[]
\ImportTok{import}\NormalTok{ numpy }\ImportTok{as}\NormalTok{ np}
\ImportTok{import}\NormalTok{ matplotlib.pyplot }\ImportTok{as}\NormalTok{ plt}
\ImportTok{import}\NormalTok{ scipy }\ImportTok{as}\NormalTok{ sp}
\ImportTok{from}\NormalTok{ scipy.io }\ImportTok{import}\NormalTok{ loadmat }\CommentTok{\#this actually enables the import of .mat}

\CommentTok{\#import matlab data}
\NormalTok{time\_domain }\OperatorTok{=}\NormalTok{ loadmat(}\StringTok{"../../zip\_meiners/behav/dsm\_l1/data/dat\_time\_domain\_plot.mat"}\NormalTok{)}
\NormalTok{spectral\_analysis1 }\OperatorTok{=}\NormalTok{ loadmat(}\StringTok{"../../zip\_meiners/behav/dsm\_l1/data/dat\_spectral\_analysis1.mat"}\NormalTok{)}
\NormalTok{spectral\_analysis2 }\OperatorTok{=}\NormalTok{ loadmat(}\StringTok{"../../zip\_meiners/behav/dsm\_l1/data/dat\_spectral\_analysis2.mat"}\NormalTok{)}
\NormalTok{windowed\_plot }\OperatorTok{=}\NormalTok{ loadmat(}\StringTok{"../../zip\_meiners/behav/dsm\_l1/data/dat\_windowed\_plot.mat"}\NormalTok{)}

\CommentTok{\#to access imported matlab data}
\NormalTok{time }\OperatorTok{=}\NormalTok{ time\_domain[}\StringTok{\textquotesingle{}t\textquotesingle{}}\NormalTok{]}
\NormalTok{voltage }\OperatorTok{=}\NormalTok{ time\_domain[}\StringTok{\textquotesingle{}u\textquotesingle{}}\NormalTok{]}
\NormalTok{the\_other }\OperatorTok{=}\NormalTok{ time\_domain[}\StringTok{\textquotesingle{}v\textquotesingle{}}\NormalTok{]}
\NormalTok{the\_other\_mod }\OperatorTok{=}\NormalTok{ np.append(the\_other[}\DecValTok{0}\NormalTok{], }\DecValTok{1}\NormalTok{)}

\NormalTok{plt.figure(}\DecValTok{1}\NormalTok{)}
\NormalTok{plt.plot(time[}\DecValTok{0}\NormalTok{], voltage[}\DecValTok{0}\NormalTok{], linewidth}\OperatorTok{=}\DecValTok{2}\NormalTok{, label}\OperatorTok{=}\StringTok{\textquotesingle{}Voltage\textquotesingle{}}\NormalTok{)}
\CommentTok{\#plt.stairs(time[0], the\_other\_mod)}
\CommentTok{\#plt.axis([min(time).any(), max(time).any()/8, 1.1*min(the\_other\_mod).any(), 1.1*max(the\_other\_mod).any()])}
\NormalTok{plt.title(}\StringTok{\textquotesingle{}Time domain plot first order $\textbackslash{}Delta \textbackslash{}Sigma$ modulator\textquotesingle{}}\NormalTok{)}
\NormalTok{plt.xlabel(}\StringTok{\textquotesingle{}Time [s]\textquotesingle{}}\NormalTok{)}
\NormalTok{plt.ylabel(}\StringTok{\textquotesingle{}Amplitude [V]\textquotesingle{}}\NormalTok{)}
\NormalTok{plt.legend()}
\NormalTok{plt.grid()}
\end{Highlighting}
\end{Shaded}

\pandocbounded{\includegraphics[keepaspectratio]{guide_files/figure-pdf/cell-3-output-1.png}}




\end{document}
